%%%%%%%%%%%%%%%%%%%%%%%%%%%%%%%%%%%%%%%%%
% Weekly Report 
% LaTeX Template
% Version 1.3 (26/10/2018)
% Modified by
% Enes TAŞTAN
% Erdem TUNA
% Halil TEMURTAŞ
%%%%%%%%%%%%%%%%%%%%%%%%%%%%%%%%%%%%%%%%%
%
%----------------------------------------------------------------------------------------
%	PACKAGES AND OTHER DOCUMENT CONFIGURATIONS
%----------------------------------------------------------------------------------------
\documentclass[a4paper,12pt]{article}
%-----packages------
\usepackage[a4paper, total={6.2in, 8.5in}, headheight=110pt]{geometry}
\usepackage[english]{babel}
\usepackage[utf8x]{inputenc}
\usepackage{amsmath}
\usepackage{graphicx}
\usepackage[colorinlistoftodos]{todonotes}
\usepackage{gensymb} % this could be problem
\usepackage{float}
\usepackage{fancyref}
\usepackage{subcaption}
\usepackage[toc,page]{appendix} %appendix package
\usepackage{xcolor}
\usepackage{listings}


\usepackage[export]{adjustbox}

\usepackage{xspace}
\usepackage{amssymb}
\usepackage{nicefrac}
\usepackage{gensymb}
\usepackage{fancyhdr}
\usepackage{lipsum}  % for lipsum
\usepackage[final]{pdfpages}  % pdf include
\usepackage{array} %allows more options in tables
\usepackage{pgfplots,pgf,tikz} %coding plots in latex
\usepackage{capt-of} % allows caption outside the figure environment
\usepackage[export]{adjustbox} %more options for adjusting the images
\usepackage{multicol,multirow,slashbox} % allows tables like table1
%\usepackage[hyperfootnotes=false]{hyperref} % clickable references
\usepackage{epstopdf} % useful when matlab is involved
%\usepackage{placeins} % prevents the text after figure to go above figure with \FloatBarrier 
%\usepackage{listingsutf8,mcode} %import .m or any other code file mcode is for matlab highlighting

%-----end of packages
%\input{../../../Documents/configuration.tex}


\pagestyle{fancy}
\setlength\headheight{80pt}
\setlength{\footskip}{2.5cm}
%\fancyhead[LO,LE]{Duayenler Ltd. Şti.}
%\fancyhead[RO,RE]{October 19, 2018}
%\fancyhead[LO,LE]{\textbf{Duayenler Ltd. Şti.} \\ \textbf{Members :\\ } 
%			Enes Taştan, 2068989, 0543 683 4336 \\ 
%			Halil Temurtaş, 2094522, 0531 632 2194  		
%}
%\fancyhead[RO,RE]{
%			\textbf{XXXX, XX, 201X} \\
%			Sarper Sertel, 2094449, 0542 515 6039 \\
%			Erdem Tuna, 2167419, 0535 256 3320 \\ 
%			İlker Sağlık, 2094423, 0541 722 9573 		
%}
%\fancyhead[RO]{Sarper Sertel (05435156039),\\Enes Taştan (05436834336), Erdem Tuna (05352563320),\\Halil Temurtaş (05316322194), İlker Sağlık (05417229573)}

\begin{document}
	
\begin{center}
	\Large\textbf{30.11.2018 TAI Ziyareti}
	\end{center}

\section{Summary of the Day}

	\large\textbf{Aviyonik Ekipman Dökümanlarını Adımız Gibi Bilmemiz Lazım}

\section{To-Do}

\begin{itemize}
	
	\item Aviyonik Ekipman dökümananları detaylı okunacak
	
	\item Garminlerle ilgili Interface Connection Document? ı oku. 
	
	\item Diğer gruplarla kullanacakları sensör/elektriksel parçalar için iletişime geçilecek
		\subitem *Ne kadar akım çekilecek vs tahmini belirlemek lazım
	\item Aviyonikleri SDD seviyesinde bilmek lazım
		\subitem *Taslak dökümanlar Erdem Abiden istendi.
	\item Döküman seviyeleri (En genelden, En ayrıntılıya)
		\begin{itemize}
			\item SDD (System Design? Documentation)
			\item SID (System Interface Documentation)
			\item SRD (System Requirement Documentation)
			\item SWRD (System Wiring R.. Documentation)
				\subitem *SWRD sonrası artık kabloların CAD yerleşimi
				\subitem *Taslak çıkardıktan sonra, CAD kısmı yerleşimcilere verilebilir
		\end{itemize}			
	\item Electrical Architecture
		\begin{itemize}
			\item 3 Bus
				\subitem *Main Bus
				\subitem *Essential Bus (Battery-1) [Hürkuş'ta önde]
				\subitem *Emergency Bus (Battery-2) [Hürkuş'ta arkada]
			\item Hangi durumda hangi busdan kim ne çekecek?
		\end{itemize}		
		
	\item Circuit Breakerlar sistem parçalarının elektrik alıp almamasına karar veren mekanizma
		\subitem *Görünür yerde olacak mı?
	\item Switchler ise sistem parçalarının açılıp kapanması için
	\item Bahsi geçen diğer konular
		\begin{itemize}
			\item Harici elektrik kaynağı olacak mı? [Motor başlangıcı için vs.]
			\item \textbf{Grounding Aşırı Önemli}
			\item Pitostatic tüpün seçimi $100\%$ bizde olmasa da, kullanımı ve bağlantıları bizde
 			\item Hürkuş'ta iki bilgisayar var (1 main+1 back-up) [Birbirini yedekliyor]
			\item Radalt [sensör]
				\subitem *FSM(Frequency Shift Modulation) , Ground seviyesi ile mesafeyi ölçüyor, [Hürkuş'ta ELT anteninin hemen arkasında]
			\item ELT anten [Hürkuş'ta kanatların arkasında, aşağı tarafta]
			\item VOR (Sadece açı bilgisi sağlıyor, açı+mesafe bilgisi veren diğer askeri sensörü kullanmamıza gerek yok)
			
			\item WOW (Weight of Wheels?) iniş takımlarının açılıp kapandığını anlayan manyetik sensör
			\item Hürkuş'ta digital göstergelerin yanında analog G-metre eklenmiş
			\item Warning/caution sistemimiz olacak mı? [Hürkuş'ta var]
			\item Aviyonik iletişimi için, 429 [poit-to-point] iletişim protokolünü incele [1553 kullanmıyoruz] 
		\end{itemize}
	
\end{itemize}



%\lipsum[1-5]

\end{document}

%----samples------
%\begin{itemize}
%\item Item
%\item Item
%\end{itemize}

%\begin{figure}[H]
%\center
%\setlength{\unitlength}{\textwidth} 
%\includegraphics[width=0.7\unitlength]{images/logo1}
%\caption{\label{fig:logo}Logo }
%\end{figure}

%\begin{figure}[H]
%	\setlength{\unitlength}{\textwidth} 
%	\centering
%	\begin{subfigure}{.5\textwidth}
%  		\centering
%  		\includegraphics[width=0.48\unitlength]{images/logo1}
%  		\caption{\label{fig:logo1}Logo1 }
%	\end{subfigure}%
%	\begin{subfigure}{.5\textwidth}
%  		\centering
%		\includegraphics[width=0.48\unitlength]{images/logo2}
%  		\caption{\label{fig:logo2}Logo2}
%	\end{subfigure}
%\caption{\label{fig:calisandegree} Small Logos   }
%\end{figure}
	
%\begin{table}[H]
%  \centering
% 
%    \begin{tabular}{c|c|c}
%       $$A$$ & $$B$$ & $$C$$ \\ \hline
%       1 & 2 & 3  \\ \hline
%       2 & 3 & 4  \\ \hline
%       3 & 4 & 5  \\ \hline
%       4 & 5 & 6  
%      
%  \end{tabular}
%  \caption{table}
%  \label{tab:table}
%\end{table}
	
%\begin{table}[H]
%  \centering
% 
%    \begin{tabular}{c|c|c}
%       \backslashbox{$A$}{$a$} & $$\specialcell{ Average deviation \\ after subtracting out the  \\ frequency error }$$ & $$C$$ \\ \hline
%       \multirow{2}{*}{1} & 2 & 3  \\ \cline{2-3}
%        & 3 & 4  \\ \hline
%       3 & \multicolumn{2}{c}{4}  \\ \hline
%       4 & 5 & 6  
%      
%  \end{tabular}
%  \caption{table}
%  \label{tab:table}
%\end{table}
%-----end of samples-----